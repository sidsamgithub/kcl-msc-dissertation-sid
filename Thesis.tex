%&latexf
\documentclass[]{kclthesis}

%===========================================================
% Change below accordingly and remove``\red{}''
%===========================================================

% TODO: WILL NEED TO MAKE THIS INTO A TABLE TO ALIGN IT PROPERLY

\department{Informatics} %use the right department
\modulecode{7CCSMPRJ} %Use the right module code
\submissiontitle{Individual Project Submission 2024 - 2025}
\author{Siddharth Kishor Samarth}
\studentnumber{K24012370}
\programme{MSc. Advanced Computing}
\title{GluCORRECT - Harnessing Artificial Intelligence to scrutinize Hypoglycemia in hospitalised patients with diabetes to classify, anticipate and analyse hypoglycemic episodes [Knowledge Exchange Project with NHS England]}
\supervisor{Dr. Rita Borgo}
\wordcount{\red{==== Word count goes here ====}}


\ReleaseProject{1} %Replace 0 by 1 if release project; Replace 0 by 2 if not release project

%===========================================================


% nomenclature 
\usepackage[intoc]{nomencl} 
%\makenomenclature
\makeindex
% glossaries
\usepackage[toc, acronym]{glossaries} 
\usepackage{float}
\usepackage{array} % for table content col / row alignment
\usepackage{hyperref}
\usepackage[utf8]{inputenc}
\usepackage{graphbox}
\usepackage{hyperref}
\usepackage{ragged2e} % for justified text
\usepackage{soul} % highlighting text
% \usepackage[none]{hyphenat} % to prevent latex from hyphenating words at the end of lines - used /hyphenpenalty (below) instead
\tolerance=1 % to prevent latex from hyphenating words at the end of lines
\emergencystretch=\maxdimen % to prevent latex from hyphenating words at the end of lines
\hyphenpenalty=10000 % to prevent latex from hyphenating words at the end of lines

\hbadness=99999 % get rid of overflow warning in vscode, can remove later

\linespread{1}
\newfam\msbfam
\def\Bbb#1{\fam\msbfam\relax#1}

\newtheorem{theorem}{Theorem}[section]
\newtheorem{exa}{Example}[section]
\newtheorem{corollary}[theorem]{Corollary}
\newtheorem{lemma}[theorem]{Lemma}
\newtheorem{proposition}[theorem]{Proposition}

\theoremstyle{definition}
\newtheorem{definition}[theorem]{Definition}
\newtheorem{remark}[theorem]{Remark}
\newtheorem{notation}[theorem]{Notation}
\newtheorem{assumption}[theorem]{Assumption}
\newtheorem{conjecture}[theorem]{Conjecture}

\newcommand{\ind}{1\hspace{-2.1mm}{1}} %Indicator Function
\newcommand{\I}{\mathtt{i}}
\newcommand{\D}{\mathrm{d}}
\newcommand{\E}{\mathrm{e}}
\newcommand{\RR}{\mathbb{R}}
\newcommand{\sgn}{\mathrm{sgn}}
\newcommand{\atanh}{\mathrm{arctanh}}
\def\equalDistrib{\,{\buildrel \Delta \over =}\,}
\numberwithin{equation}{section}
\def\blue#1{\textcolor{blue}{#1}}
\def\red#1{\textcolor{red}{#1}}

\setcounter{tocdepth}{5}
\setcounter{secnumdepth}{5}

% customize your general setup here
%\title{\red{Project title goes here}}
%\author{\red{Your name goes here}}


\begin{document}
\pagenumbering{gobble}


%%%%%% depends what you like, might try out the other frontpage as well
\maketitle 		% official styled layout
\maketitleTwo 	% adapted layout which looks nicer from my point of view.

%%%%%% empty page after main page.
\newpage
\thispagestyle{empty}
\mbox{}
\newpage
%%%%%% Acknowledgements, Abstract, Nomenclature
% \include{contents/TemplateDescriptions} %Remove this line after reading
% \red{The content of ``Acknowledgements'' is in ``{\textbackslash}contents{\textbackslash}acknowledgements.tex''}

\mbox{}\newline\vspace{10mm} \mbox{}\LARGE
%
{\bf Acknowledgements} \normalsize \vspace{5mm}

\par I would like to express my sincerest gratitude towards my project supervisor, Dr. Rita Borgo, for her invaluable advice and consistent 
direction throughout the course of this project. Her mentorship and ideas have been instrumental in shaping the development of this work, 
leading to its successful completion.

\par I am also deeply thankful \& appreciative of my industry advisor, Dr. Piya Sen Gupta, for providing the dataset that has served as the 
foundation of this work. Her contributions have significantly enhanced the practical relevance and quality of this project.

\par Ultimately I would like to thank my friends and my parents, especially my dad, without whose sacrifices I would not be where I am today.
JELLO WORLD THIS IS TEST OF LATEX IN VSCODE

\red{The content of ``Abstract'' is in ``{\textbackslash}contents{\textbackslash}abstract.tex''}

\section*{Abstract}

    \begin{flushleft}
    
    \textbf{Variant 4 - Develop a weighted score and design score to predict episode of hypoglycaemia before it occurs.} \\ 

    \vspace{10pt}
    
    \par This research project serves as a foundation and proof-of-concept to aid GSTT in pre-emptively reducing hypoglycemia and its episodes within hospitalised patients, by utilising statistics \& machine learning techniques. I go on to identify the significant factors responsible for hypoglycemia, and explore how they can be utilized to create a risk score, to classify patients with severe or recurrent severe hypoglycemia, and in conclusion, exhibit my findings with potential ways of applying them in practise in hospitals.
    
    
    \end{flushleft}
    
% \begin{itemize}
% 	\item A brief introduction to the project objectives
% 	\item A brief description of the main work of the project
% 	\item A brief description of the contributions, major findings, results achieved and principal conclusion of the project
% \end{itemize}

 All abbreviations and symbols used in the report must be listed and defined in alphabetic order.

\section*{Nomenclature}

\begin{flushleft}
\begin{minipage}{1\textwidth}
    	\centering
        \def\arraystretch{1.25}%
    	\begin{tabular}{ll}
			GSTT  & Guy's and St Thomas' NHS Foundation Trust \\
			HQIP & Healthcare Quality Improvement Partnership \\
			NCAPOP & National Clincal Audit \& Patient Outcomes Programme \\
			NDA   & National Database Audit  \\
    		NDISA & National Diabetes Inpatient Safety Audit \\  
    		NHS & The publicly funded healthcare system of the United Kingdom, \\ 
            & the National Health Service. \\
      
                
    	\end{tabular}
\end{minipage}

\end{flushleft}

\noindent\rule{8cm}{0.4pt} \\ % temporary horizontal rule
$a$ \qquad The number of angels per unit area\\
$A$ \qquad The area of the needle point\\
$c$ \qquad Speed of light in a vacuum inertial frame\\
$h$ \qquad Planck constant\\
LMI	\qquad Linear Matrix Inequalities\\
$N$ \qquad The number of angels per needle point
%%%%%% Table of contents

\pagenumbering{roman}
% \setcounter{tocdepth}{4} % default
\setcounter{tocdepth}{3} % to hide subsubsections
\tableofcontents
\newpage
%%%%%%%%%%%%%%%%%%%%%%%%%%%%%%%%%%%%%%%%%%%%%%%%%%%%
\include{contents/figures_tables}
\fancyhead{}
\fancyfoot{}
\pagestyle{fancy} 
%\fancyhead{\sffamily\small \thepage}
%\fancyhead{\sffamily\small \nouppercase{\rightmark}}
\fancyhead[RO,LE]{\sffamily\small \thepage}
\fancyhead[LO,RE]{\sffamily\small \nouppercase{\rightmark}}
\renewcommand{\headrulewidth}{0.4pt}
\renewcommand{\footrulewidth}{0.0pt}
%%%%%%%%%%%%%%%%%%%%%%%%%%%%%%%%%%%%%%%%%%%%%%%%%%%%

%%%%%% Main content
\pagenumbering{arabic}

%Insert if necessary if more chapters are needed by creating a new .tex file in the folder ``contents'' and use \include{filename} to include the new chapter.
%Modify the section headings if necessary
%Insert more sections if necessary
\section{Introduction}
% It provides the background and context of the work.

\begin{flushleft}
\end{flushleft}

\subsection{Clinical Overview}\label{sec:clinicalOverview}
\par{ \noindent Hypoglycaemia (also known as a “hypoglycaemic episode” or a “hypo” for short) is the condition that occurs when the human body’s blood glucose (sugar) level drops below the normal healthy range of 4.0 to 6.0 mmol/L. While it can affect anyone, it is most common in diabetic individuals who are prescribed drugs like insulin or metformin to inhibit glucose. Hypoglycaemic events are relatively simple and straightforward to resolve, but they need to be treated immediately to avoid serious damage to the brain and heart as a result of loss of consciousness or arrhythmias. High-sugar consumables are generally effective in correcting mild cases and are commonly recommended for immediate treatment, but severe cases of hypoglycaemia such as when the person is unconscious or having a seizure can only be resolved with an urgent, immediate glucagon injection to prevent them from deteriorating into a coma (or in rare cases, even leading to death). }


\vspace{10pt}
\par{\noindent To underscore how and why this matters, diabetes is one of the most significant and expensive long-term health conditions faced by the NHS, with recent figures from Diabetes UK suggesting that over 5.8 million people in the UK are living with diabetes, regardless of a formal diagnosis. It is estimated to cost the NHS over £10.7 billion a year, approximately 10\% of its entire annual budget, which could go up to £18 billion by 2035  \cite{diabetescosts}. A stark finding is that almost 60\% of this cost (around £6.2 billion) is spent on treating the largely preventable complications of diabetes, such as heart attacks, strokes, blindness, and so on, including hypoglycaemia  \cite{preventablediabetescosts}. Hypoglycaemic instances make up a major component of these preventable costs, mainly accounting for the emergency, ambulance, and acute care expenses associated with diabetes. The Local Impact of Hypoglycaemia Tool (LIHT) suggests that hypoglycaemia can cost up to £2,195 per episode, possibly increasing substantially with a longer stay in hospital  \cite{lihttool}, and it is estimated that there are up to 100,000 ambulance callouts annually according to the Diabetes Research and Wellness Foundation (DRWF)  \cite{diabetesemergencycallouts}. DRWF’s study hinted that 1 in 10 individuals that experience a severe hypo (meaning requiring medical intervention or resuscitation) have considerable chances of another one within a fortnight.}

\subsection{Background} 
\par{\noindent After introspective analysis supported by information from the National Diabetes Inpatient Safety Audit (NDISA) it has been recognized that severe hypoglycaemia and recurrent severe hypoglycaemia have been occurring relatively frequently across GSTT medical facilities. The NDISA forms part of the National Diabetes Audit (NDA), and it maintains that “The prevalence of diabetes continues to increase. In England between 2017-18 and 2021-22 prevalence of type 1 diabetes went up from 248,240 to 270,935 and the prevalence of type 2 and other diabetes from 2,952,695 to 3,336,980”, as of 2022  \cite{nationaldiabetesaudit22}. }

\vspace{10pt}
\par{\noindent GSTT administers upwards of 500,000 point-of-care glucose tests (POCT) annually, in addition to kidney function and glycated haemoglobin (HbA1c) tests as well. The Trust also possesses blood glucose / ketone data with additional linked data including demographics, dates of admission and discharge, patient as well as family history and current or previous medications. They have two major kinds of patient records, inpatient records for patients that have to stay over the course of one or multiple nights (for example, in case of surgeries or for long term care), and outpatient records where the patient doesn’t require overnight stay. The Trust manages all of this data through their electronic health record management system called Epic, and facilitates patient access to their own records through the MyChart web application.}

\vspace{10pt}
\par{\noindent Hypoglycaemia is a frequent complication amongst inpatients having complex health conditions, especially within those in intensive care settings that have been / are critically ill due to advanced diseases or comorbidities, or in patients following major surgical interventions. The Trust is undertaking proactive measures to identify and mitigate the risk of hypoglycaemic episodes at an early stage, to support better planning, reduce healthcare costs, efficiently allocate hospital resources and also schedule operations optimally. The ideal way to assess risk would along the lines of developing tools to predict individualized risk scores for inpatients after considering all relevant factors. However, this presents a herculean task due to the sheer volume and complexity of factors involved, compounded by the challenges of producing reliable results even within small populations — such as those in remote areas — while also adhering to legal and governmental regulations: 
\begin{enumerate}
	\item Weighing up the risk of hypoglycaemia depends upon numerous aspects such as lifestyle, renal function, recent food intake, blood glucose history and current medication to name a few, making this a highly complicated modelling problem. In addition to this, patients differ widely in age, comorbidities, ethnic factors and even insulin sensitivity. This variability makes it a formidable challenge to develop a model that is generalizable, dependable and unbiased.
	\item Any such analytical tool in the vicinity of patient healthcare requires medical evaluation and approval, validation trials, governance oversight as well as ethical considerations. Even a good model may fail if it does not fit the clinical workflow. Initial skepticism towards AI, the effort required to train staff, defining clear responsibilities and limits of liability, and rehearsing procedures or plans of action for every possible scenario will all produce appreciable organizational inertia.
\end{enumerate}
}

\vspace{10pt}
\par{\noindent Successfully implementing even a small-scale solution, within GSTT to begin with, would be a significant strategic breakthrough that serves as a foundational model which other NHS trusts or institutions could adapt and build upon. This positions this research initiative which is a Knowledge Exchange Project (KEP) with Guy’s \& St.Thomas’ NHS Foundation Trust, an indispensable constituent of London's healthcare system, as a valuable and worthwhile research endeavour.}


\subsection{Aims and Objectives} 
This research project has the following objectives:

\begin{itemize}
	\item \textbf{To extract insights from provided dataset for the given time period and population.} GSTT has expressed a strong interest towards gaining a deeper understanding of their inpatient population. The dataset they have provided includes demographic details, length of hospital stay, and ward information in addition to the main clinically relevant variables such as glycated haemoglobin levels, renal function measurements, patient age and so on. This enables a comprehensive, multifaceted analysis. The knowledge gained from this study, such as identifying which hospital wards have more vulnerable or at-risk patients, will be used to enhance staff training, in turn improving both future admissions routines as well as post-discharge support for patients. Every observation, regardless of scale, holds potential to refine hospital processes and operating procedures.
	\item \textbf{To identify the main influencing / contributing factors for hypoglycaemia and develop a weighted risk score to predict episodes (Variant 4 KEP).} The Trust is establishing and implementing measures to "pre-assess" inpatients to evaluate their risk of a hypoglycaemic episode, which will allow medical professionals to design protocols and policies to prevent episodes from occurring as well as take early remediative action as soon as possible to resolve an episode should it occur. I aim to find data-backed values for the key features responsible for hypoglycaemia, through statistical tests and machine learning algorithms, in order to formulate a risk score. This risk score can then be applied in hospital to determine the best course of early action or precautions to take based on the patient's reason for being admitted.
\end{itemize}

\subsection{Report Structure}
\par{ \noindent Section 2 contains a comprehensive, detailed review of similar research carried out by other universities, teaching hospitals and medical facilities including references to relevant medical literature. I have compared and contrasted datasets used, approaches taken and results obtained.}
\par{\noindent Section 3 delves deeper into the dataset provided by GSTT, elaborating on the raw features provided and those that were derived from them for analysis.}
\par{\noindent Section 4 (Methodology \& Implementation) outlines the statistical and mathematical theory behind the concepts used for analysis, ranging from machine learning algorithms to hypothesis testing methods.}
\par{\noindent Sections 5 (Main Results) onwards discuss the main research executed within the project and deliberates on the results achieved}
\par{\noindent Section 6 (Ethical Professional Legal Social issues) }
\par{\noindent Section 7 (Conclusion and Applicability) }



\subsubsection{Dissertation Length}
This dissertation comprises a total of \red{\@wordcount XXXX} words excluding references and appendices.

% ================================================ TEMPLATE LEFTOVERS for ref ==========================================================

% \subsection{Aims and Objectives} 
% % The problems and project objectives should be stated comprehensively. The motivations of the project should be presented. The techniques and approaches used to deal with the problem should be stated with justifications, and the contributions and main results achieved should be stated clearly. The structure of the report can be described briefly at the end see \autoref{sub:background}.



% \subsubsection{Dissertation Length}
% 	The dissertation should be less than 15000 words.
	
% \paragraph{Dissertation Length 2} ~\newline
% 	Refer to KEATS for suggested structure
	
% \noindent \subparagraph{More subsections}~\newline
	
	

% \subsection{Background and Literature Survey} \label{sub:background}
%  It gives an overall picture about the work with a clear review of the relevant literature.  The background of the project should be given.  What have been done to deal with the problem should be stated clearly.  The pros and cons of various existing algorithms and approaches should be stated as well.  Differences between your proposed method and the existing ones should be briefly described. It is important to make sure that the discussion is structured and coherent; the key issues are summarised; key and relevant references are used critically analysed and the literature is covered comprehensively.

% The following links may help on literature review:
% \begin{itemize}
% 	\item \textbf{IEEE Xplore digital library} (\hyperref[http://ieeexplore.ieee.org]{http://ieeexplore.ieee.org/}): a resource for accessing IEEE published scientific and technical publications (You must be with King's network to get access to the digital library)
% 	\item \textbf{ScienceDirect.com} (\textbf{ScienceDirect.com} \hyperref[http://scienceDirect.com]{http://scienceDirect.com}): an electronic database offering journal papers not published by IEEE (You must be with King's network to get access to the database)
% \end{itemize}

% \subsection{Insert More Sections if Necessary}
% \red{The content of ``Background'' is in ``{\textbackslash}contents{\textbackslash}background.tex''}

\section{Literature Survey \& Review} 

\subsection{The Diabetes Management Tightrope}

\noindent Managing diabetes often draws parallels with walking a metabolic tightrope. On one side lies the danger of hyperglycaemia and its associated \textit{long-term complications} such as diabetic retinopathy(damage to blood vessels in the eye leading to blindness) or renal function impairment (diabetic nephropathy), while on the other lies the \textit{immediate peril of hypoglycaemia}. For the longest time, clinicians have helped patients navigate this delicate balance, by utilising methods or practises that show where they are, but not necessarily where they are going, in terms of blood glucose measurements. Such a reactive approach with little account for anticipative elements has made hypoglycaemic episodes an unavoidable consequence of striving for tight glycaemic control. 

\vspace{5pt}
\noindent This "tightrope" extends beyond just taking efforts to stay safe, requiring diabetics and patients to perform constant risk assessments in everyday life. UK driving law from the DVLA mandates a blood glucose reading of above 5.0mmol/L with a repeat test every two hours for longer journeys in order to be considered safe to drive. Patients are required to carry a "hypo kit" with fast-acting glucose at all times. The worry about having episodes in public and having to rely on the awareness of strangers or being a hindrance to social situations is a constant concern. As discussed \hyperref[sec:clinicalOverview]{here previously} this also places notable financial strain on NHS resources through emergency or ambulance costs.

\vspace{5pt}
\noindent With scientific and technological progress that inevitably comes with time, comes the promise of a potential safety net: the ability to anticipate or foresee signs of an episode before they present. The dangerous tightrope walk can then be transformed into a manageable path, with the help of predictive systems built with advanced sensing technologies and computational power, that can assist patients to take pre-emptive action. This review will chart the progress in this field, understanding the methodologies and ideas that have been applied to forecast hypoglycaemic events, from early tracking methods to highly optimized modern algorithms.


\subsection{Methods Of Monitoring Blood Glucose Over Time}

\noindent The success rate of recognising patterns in blood glucose has been vastly upgraded through the years, spurred by both technological and procedural refinements in the monitoring of blood glucose, but it has been an arduous journey to get here. The earliest methods of testing blood glucose involved urine tests, where chemical reagents like Benedict's solution were used which changed colour in the presence of sugar. Such methods were only qualitative and retroactive - they offered no actionable information as they confirmed that blood glucose had been elevated at the time of the test or in the recent past. The first blood glucose meters did not appear until the 1960s, were large and cumbersome to work with, and were mainly found in clinics. Smaller, portable meters became available around the 1970s - 1980s yet still all such meters had to be used repetitvely throughout the day, offering only a snapshot of blood glucose level in time without any means of showing a trend, stability or lack thereof. 

\vspace{5pt}
\noindent The introduction of Continuous Glucose Monitoring (CGM) solutions from 2005 onwards completely revolutionized diabetic healthcare, allowing measurements to be taken effortlessly through sensors attached to the body. Medtronic and Dexcom's devices released after 2015 with highly improved sensor accuracy and user-friendliness even allowed connecting to smartphones and automatic insulin pumps, which could automatically stop insulin delivery if the patient did not respond to an alarm. Today's CGM's are even more cutting-edge, in that they continuously transmit data to a receiver. They are smartphone-app based for ease of use, offering enhanced features to show the trend and speed of glucose changes, alarms and alerts to proactively warn users, as well as sharing data with family members or doctors for remote monitoring and emergency handling \cite{evolutionOfCGM}. 

\vspace{5pt}
\noindent In modern times, there is huge amounts of data available to probe into, but the challenge lies in finding the correct relevant features, at the correct level of granularity as well as the right distribution, as medical data is exceptionally rarely obtained in balanced form. The availability of rich continuous streams of data from CGMs has stimulated additional research dedicated to developing and applying algorithms to both forecast hypoglycaemic events as well as identify outliers or patterns within the data, and I delve into this next.

\subsection{Review Of Relevant Literature and Similar Research}

\noindent A substantial body of literature now exists on the development of models for predicting hypoglycaemia in various settings. In many scenarios, a significant proportion of the research focuses on optimizing predictive accuracy of models. For instance H. Yang et al. in 2022 \cite{yangHaoJMIRMedicalInformatics} have used electronic health records(EHR) of patients admitted to West China Hospitals to develop a predictive model based on laboratory derived biomarkers (like lipoproteins, creatinine, globulin etc.).  A similar research to this was undertaken by S. Mantena et al. \cite{sMantenaCriticallyIll}  but on the publicly-available eICU Collaborative Research v2.0 database (eICU-CRD) that holds de-identified data for 200,000+ admissions in 553 ICUs across the USA. In reinforcement to this, UK-based studies have also been executed by Y. Ruan et al. \cite{yRuanOxfordHypoRiskPred} on four years worth of EHRs provided by Oxford University Hospitals NHS Foundation Trust in which they compared the performance of eighteen different predictive models based on demographic, laboratory, vital signs and previous medication predictors.  

\vspace{5pt}
\noindent A significant commonality was observed in all the three research works, which I have incorporated into my approach as well - the emergence of XGBoost as the best predictive model with highest Area Under Receiver Operating Characteristic Curve(AUROC). Even though all studies produce excellent results in terms of predictive performance, they are not primarily aimed at finding the critical values or "turning points" of the predictors at which predictions / classifications change, which is my objective for this project that also lines up with GSTT priorities.

but almost all of this is based on highly controlled circumstances where a   












\begin{flushleft}
Various kinds of different prediction models have already been devised and developed for predicting hypoglycemia. Yi Wu and others have systematically compared, and evaluated the applicability of models in clinical practice in a paper in Biological Research for Nursing[1] 
where it was found that the major predictors were age, HbA1c, history of hypoglycemia, and insulin use. Lin Yang, Zhiguang Zhou have carried out similar research in the Frontiers in Public Health journal[2] uncovering risk factors that could possibly lead to hypoglycemic 
events, after employing various data driven models based on ML techniques such as neural networks, autoregressive / ensemble learning and such. \\ \vspace{5pt}

In silico proof of concept studies like the one from Zecchin[3] have also been researched to investigate how continuous glucose monitoring short-term glucose prediction algorithms could be exploited to recognise the run up to hypoglycemic episodes, allowing the patient to 
take appropriate countermeasures to mitigate events. They found that there was a significant reduction in both the time spent in a hypoglycemic event as well as the number 
of hypoglycemic events. \\ \vspace{5pt}

\end{flushleft}
\section{Dataset}

\par{\noindent

    Our industry advisor from GSTT has graciously provided a year's worth of data in multiple .xlsx files, which have been combined into one for the purposes of analysis and research for this project. The \textbf{raw fields provided} within the data were:  

    \begin{itemize}
        \item \textbf{UniqueID:} Unique identifier for the patient and test, which is just a number. Meant to identify same patients (not personally) when considered together with Order Time, Order Date and Age, as same patients can have multiple blood glucose tests during their stay. 
        \item \textbf{Order Date:} The date when the glucose measurement was ordered or taken 
        \item \textbf{Order Time:} The timestamp at which the glucose measurement was ordered or taken
        \item \textbf{Inpatient Admission Date:} The date at which the patient was admitted into the medical facility
        \item \textbf{Discharge Date:} The date the patient was discharged from the medical facility
        \item \textbf{Length of Stay:} The amount of time the patient has spent in the medical facility in days and hours (for eg. "5d 6h")
        \item \textbf{Ward:} The ward that the glucose measurement was taken in, usually matches the ward that the patient was admitted to
        \item \textbf{Last Lab Test Results:} The result of the glucose measurement in mmol/L. Most values in this column are of the format "Manual blood glucose: 8.70 mmol/L" or "POCT Glucose Blood Manually: 2.7 mmol/L".
        \item \textbf{Age:} Age of the patient at the time of measuring blood glucose in years
        \item \textbf{Ethnicity:} Specific ethnicity of the patient, values ranging from "South American - Columbian" to "Black or Black British - Nigerian" to "Other" or even missing.
        \item \textbf{Gender Identity:} Gender of the patient.
        \item \textbf{HbA1c:} Numerical value of HbA1c in mmol/mol.
        \item \textbf{HbA1c Date:} Date the HbA1c test was done for that patient
        \item \textbf{eGFR:} Estimated Glomerular Filtration rate, which is a measurement of how well the kidneys are functioning. This is a percentage from 0 to 90, with anything 91 and over displayed by the NHS electronic health record system (Epic) as "\textgreater91" because an eGFR of 91 percent and above indicates healthy renal function.
        \item \textbf{eGFR Date:} The date the eGFR test was conducted. 
    \end{itemize}

    
    \vspace{10pt}
    The variables \textbf{derived from these features} were: 
    
    \begin{itemize}
        \item \textbf{Age\textunderscore Range:}  Categorical variable to store the age category of the patient based on their age to aid in visualisation. Possible values for this column are: ``Young (1 to 25)'', ``Adult / Middle Aged (26-50)'', ``Older Adult / Old (51-75)'' and ``Elderly (76-100)''

        \item \textbf{Has\textunderscore Hypoglycemia:} Binary variable to store whether the patient has hypoglycemia. A glucose measurement of 4mmol/L and below means the patient is hypoglycemic and has 1 in this column, 0 otherwise. 

        \item \textbf{Glycemia\textunderscore Type:} Categorical variable to store the type of glycemia based on the patient's glucose measurement. \textbf{For the purposes of this project, the classes we have been instructed to use are (all units in mmol/L):} \newline
            1. ``Severe Hypoglycemia'' - for blood glucose values 2.2 and below \newline
            2. ``Hypoglycemia''- for blood glucose values from 2.3 to 4 both inclusive \newline
            3. ``Target Range''- for blood glucose values from 4.1 to 11 both inclusive \newline
            4. ``Hyperglycemia''- for blood glucose values above 11 \newline

        \item \textbf{eGFR\textunderscore Category:} Categorical variable that shows how serious the loss of kidney function is, based on the eGFR percentage. The possible values for this column are: \newline
            1. ``eGFR less than 20 - Kidney Failure'' - for eGFR less than or equal to 20\% \newline
            2. ``eGFR between 20 \& 40 - Critical Loss of Kidney Function''- for eGFR above 20\% but less than or equal to 40\% \newline
            3. ``eGFR between 40 \& 60 - Significant Loss of Kidney Function''- for eGFR above 40\% but less than or equal to 60\% \newline
            4. ``eGFR between 60 \& 80 - Moderate Loss of Kidney Function''- for eGFR above 60\% but less than or equal to 80\% \newline
            5. ``eGFR between 80 \& 90 - Minor Loss of Kidney Function''- for eGFR above 80\% but less than or equal to 90\% \newline
            6. ``eGFR above 90 - Normal kidney function''- for eGFR above 90\% (data has been processed to only include "91" for this class as healthy eGFR is 91\% and above). \newline

        \item \textbf{Wider\textunderscore Ethnic\textunderscore Group:} Categorical variable to store the overarching ethnic group based on the one specified in the ethnicity column, as that had a total of 57 unique values. Possible values are: ``Unknown or Not Stated'', ``White'', ``Mixed'', ``Asian or Asian British'', ``Black or Black British'' and ``Other Ethnic Groups''.

    \end{itemize}

    \vspace{10pt}
    \noindent Note that columns obtained after cleaning the original data to extract a numerical value (such as blood glucose) have been omitted for brevity.

}

% \par {
%     Dr. Piya Sen Gupta from GSTT provided us with a sample dataset for an initial exploratory data analysis, with another more comprehensive dataset for predictions to be provisioned later. This sample data contained glucose and other test details for around 580 patients, with the following columns: 
%         % PatientID (CodePatientID 1_588) - A unique identifier for each patient (from 1-588)
% TestDate / TestTime - Timestamp information, for tracking when tests were conducted.
% Test ID - Identifies specific medical tests performed.
% Facility / Location - Information about where the test was conducted, useful for geographical or institutional analysis.
% Device name - The device used for measurement, which can impact data accuracy.
% Value - the value of blood glucose in mmol/L.
% U-o-M (Unit of Measurement) - Defines the units in which values are recorded. This was mmol/L for every row (every value in the column).  
% AdmitAgeNBR - The patient's age at hospital admission, which can affect hypoglycemia risk.
% AdmitDTS / DischargeDTS - Admission and discharge timestamps, useful for analyzing hospitalization duration and outcomes. 
% % mins_since_admit - Time elapsed since admission, useful for tracking glucose fluctuations during hospitalization.
% GenderDSC - Patient’s gender, which can influence hypoglycemia risk.
% EthnicDSC - Ethnicity descriptor, which may correlate with genetic predispositions to hypoglycemia.
% % EGFR_Value / EGFR_Date - Estimated Glomerular Filtration Rate, an indicator of kidney function, which is important as kidney disease can affect glucose metabolism.
% % HbA1c_Value / HbA1c_Date - Hemoglobin A1c, a long-term blood glucose control marker. 
% % Glucose_Value / Glucose_Date - Blood glucose levels, directly related to hypoglycemia detection.
% % Weight - Body weight, which influences insulin sensitivity and hypoglycemia risk.
% }
\section{Methodology and Implementation}

	\subsection{Research Strategy and Approach}
	\noindent Before receiving the dataset, I have conducted an exhaustive investigation of the clinical landscape surrounding hypoglycaemia as a health condition, including studying the situations in which it commonly occurs, both in hospital settings as well as in public or everyday life. I have scrutinized a plethora of factors contributing to hypoglycaemia, including associated medicines (even conflicting medications), at-risk patient profiles, habits and lifestyles, dosing errors (both excessive as well as insufficient (\textit{insulin})), missed meals and even alcohol consumption. This has allowed me to better assess the quality of the incoming dataset and the relevance of its features. Upon requesting additional information regarding current patient medication and alcohol intake as it was not provided originally, GSTT advised that this data is unavailable because of its inconsistent self-reported nature and due to restrictions under their information governance policy that permits access to only the data deemed necessary for the project's scope.

	\vspace{10pt}
	\noindent After receiving the data, I have thoroughly preprocessed it to ensure it was suitable for meaningful analysis. This included addressing data type mismatches, deriving variables to aid in visualisation and understanding, and performing necessary imputations using appropriate methods. Duplicate and missing records were handled, categorical variables were encoded to make them compatible for predictive modelling, data validity and consistency checks were enacted to confirm that values were in expected ranges (for e.g. the glucose value field), normalization was carried out where necessary. These steps were necessary to lay a strong foundation for the subsequent application of statistical tests and machine learning models. The full preprocessing workflow is depicted in \autoref{fig:datapreprocessingwf} below. \
	
	\vspace{10pt}
	\noindent Following this, my focus was on exploratory data analysis to spot any anomalies or patterns near the surface. After devising research questions around the dataset, I have generated a collection of plots through Python's widely used seaborn library that I describe in detail in \hyperref[sec:mainResults]{the main results section}, which shed light on the prevalence of hypoglycaemia across various different scenarios. Special attention was paid to drawing comparisons between hypoglycaemic and non hypoglycaemic patients, in alignment with GSTT's interests that they had clarified in the project's early stages.

	\begin{figure}[H]
		\centering
		\includegraphics[height=8in]{figures/dataPreprocessingWf.png}
		\caption{Data preprocessing workflow}
		\label{fig:datapreprocessingwf}
	\end{figure}

	\subsection{Dealing With Imbalanced Data}
	augmentation / sampling

	\subsection{Machine Learning Theories}
	Decision tree
	random forest 
	grid search cv 
	conditional sampling 
	xgboost

	% It presents and justifies the methodology used to deal with the problem and describes in detail the implementation procedures. The background theory presented in the previous chapter can be recalled to support the proposed implementation. The originality, novelty and contribution are to be demonstrated with the discussion of the strengths and limitations.
\section{Main Results and Findings}\label{sec:mainResults}

\noindent \textbf{Note: Primarily due to the nature of exploratory data analysis and because of the quantity of research questions stated by GSTT, it was not possible to fit all the exploratory and modelling research done into the report because it would have certainly exceeded the word limit of 15000 words, while preventing me from explaining every salient point in the visualisations. Additionally even if made to fit within the word limit, it would have caused the report to explode in the number of pages making it difficult to maintain the reader's understanding. My code submission which is a python notebook (.ipynb file) is meant for exhibiting findings along with their supporting material, and I have documented many notes explaining the rationale behind every decision and the interpretation of every model. Please have a look through the python notebook in PyCharm or any IDE with Python and Jupyter to see the whole set of findings, the video demonstration will prove this.}

\subsection{Initial Data Pre-processing}
\noindent Raw data provided was first combined into a single file and then read into a Pandas dataframe. As part of initial cleanup, 2164 duplicate rows in the dataset were identified and dropped, as shown in \autoref{fig:dropDuplicates}. 

\begin{figure}[H]
		\centering
		\includegraphics[scale=0.8]{figures/python_code/drop_duplicate_rows.png}
		\caption{Dropping duplicate rows}
		\label{fig:dropDuplicates}
\end{figure}

\noindent Following this, the percentage of missing values was evaluated prior to carrying out feature engineering. \autoref{fig:missingValuePercentages} displays the percentage of missing data across all raw columns:

\begin{figure}[H]
		\centering
		\includegraphics[scale=0.4]{figures/python_code/percent_missing_values_in_every_column.png}
		\caption{Percentage of missing values in each column}
		\label{fig:missingValuePercentages}
\end{figure}

\noindent The column Admit Weight was rendered unusable here, because it was missing 97\% of its content. Weight values cannot be calculated or imputed accurately based on the only available related features that were Age and Ethnicity, as Weight is a unique characteristic value of a patient just like Height and BMI, and to use class-based (ethnicity and age based) averages would \textbf{mislead the analysis} and produce unreliable results, so Weight is not considered in any further study hereonwards.

\subsubsection{Feature Extraction}
\noindent Next, all the columns were thoroughly examined for the exact relevant numerical values (for instance, age was given as "57kg" in the raw data as depicted in \autoref{fig:rawDataset}.) and new columns were created to store only the extracted numerical values for analysis. After this point, the usable data resembled the structure of the cleaned data as shown in \autoref{fig:cleanedDataset}.

\subsubsection{Feature Engineering}
\noindent After having acquired clear numerical values from the raw data through the newly extracted features, additional derived features were formulated based on these values. The resulting features are listed in \autoref{sec:derivedData} along with the criteria and logic used in their construction.   

\subsection{Exploratory Data Analysis and Interpreting Model Results}

\noindent \textbf{Note: Primarily due to the nature of exploratory data analysis and because of the quantity of research questions stated by GSTT, it was not possible to fit all the exploratory and modelling research done into the report because it would have certainly exceeded the word limit of 15000 words, while preventing me from explaining every salient point in the visualisations. Additionally even if made to fit within the word limit, it would have caused the report to explode in the number of pages making it difficult to maintain the reader's understanding. My code submission which is a python notebook (.ipynb file) is meant for exhibiting findings along with their supporting material, and I have documented many notes explaining the rationale behind every decision and the interpretation of every model. Please have a look through the python notebook in PyCharm or any IDE with Python and Jupyter to see the whole set of findings, the video demonstration will prove this.}

% \vspace{5pt}
% \noindent GSTT had outlined several major research questions that they sought to understand through this analysis. The questions were shaped by their operational priorities and clinical needs, aiming to discover essential details that could improve as well as guide patient care strategies, and refine medical procedures. I have structured the subsequent analysis to address these focal points through visualisation techniques.

% \subsubsection{How many patients are in each ward, and which wards do patients spend the most time in on average?}
% \noindent GSTT wanted to examine patient flow within wards, focusing on ward assignments and length of stay. \textbf{The top 5 wards to house patients were found to be: } William Gull Ward (4773), St. Thomas Admissions Ward (2997), St. Thomas Albert Ward (2446), St. Thomas Anne ward(2062), St. Thomas Alexandra Ward(1715).  

% \noindent \begin{figure}[H]
% 		\centering
% 		\includegraphics[scale=0.5]{figures/python_code/patients_in_each_ward2.png}
% 		\caption{Number of patients in each ward}
% 		\label{fig:NumOfPatientsInWards}
% \end{figure}


% \subsection{Interpreting model results}

% \subsubsection{Decision Trees}



% \section{Math equations}
% % \textcolor{red}{This section is for demonstration of equations, figures, tables, which is not required for the report.}
% \subsection{Maths}
% \begin{equation}\label{eq:BS}
% \frac{\D S_t}{S_t} = r \D t + \sigma \D W_t,
% \qquad S_0>0,
% \end{equation}

% The equation $\sigma = m a$ follows easily~\cite{Doe11}.


% \subsection{Glossary and acronyms}

% \newglossaryentry{Linux}
% {
%     name=latexlinux,
%     description={Is a markup language specially suited for 
% scientific documents}
% }

% \newglossaryentry{lvm}
% {
%     name=lvmformula,
%     description={A mathematical expression}
% }

% \Glspl{Linux} and other Unix operating systems are better then Windows because they support \gls{lvm} out of the box~\cite{Joh11}\insertref{A ref is missing here}. 

% \subsection{Figures}
% Here is an example~\cite{JohSil05} of how to insert a picture:

% \begin{figure}[!ht]
% \centering
% \subfigure{\includegraphics[scale=0.2]{figures/Picture.eps}}
% \caption{This is the caption for the figure.}
% \label{fig:Pict}
% \end{figure}


% \begin{figure}[!ht]
% \centering
% \missingfigure{If you know there will be a figure, but you still need to create it.}
% \caption{This is the caption for the figure which is not even present.}
% \label{fig:PictMis}
% \end{figure}


% Lorem ipsum dolor sit amet, consetetur sadipscing elitr, sed diam nonumy eirmod tempor invidunt ut labore et dolore magna aliquyam erat, sed diam voluptua. At vero eos et accusam et justo duo dolores et ea rebum. Stet clita kasd gubergren, no sea takimata sanctus est Lorem ipsum dolor sit amet. Lorem ipsum dolor sit amet, consetetur sadipscing elitr, sed diam nonumy eirmod tempor invidunt ut labore et dolore magna aliquyam erat, sed diam voluptua. At vero eos et accusam et justo duo dolores et ea rebum. Stet clita kasd gubergren, no sea takimata sanctus est Lorem ipsum dolor sit amet.\todo{This is a small Todo, please take care!}

% or two side-by-side pictures:

% \begin{figure}[!ht]
% \centering
% \subfigure{\includegraphics[scale=0.3]{figures/Picture.eps}}
% \hspace{15pt}
% \subfigure{\includegraphics[scale=0.3]{figures/Picture.eps}}

% \caption{Another caption}
% \label{fig:Pict2}
% \end{figure}



% \subsection{Table}
% Lorem ipsum dolor sit amet, consetetur sadipscing elitr, sed diam nonumy eirmod tempor invidunt ut labore et dolore magna aliquyam erat, sed diam voluptua. At vero eos et accusam et justo duo dolores et ea rebum. Stet clita kasd gubergren, no sea takimata sanctus est Lorem ipsum dolor sit amet. Lorem ipsum dolor sit amet, consetetur sadipscing elitr, sed diam nonumy eirmod tempor invidunt ut labore et dolore magna aliquyam erat, sed diam voluptua. At vero eos et accusam et justo duo dolores et ea rebum. Stet clita kasd gubergren, no sea takimata sanctus est Lorem ipsum dolor sit amet\explainindetail{This needs further explanation}.
% \begin{table}[!ht]
% 	\centering
% 	\begin{tabular}{|l|rl|}
% 		\hline
% 		Something & Someother & Thing \\
%   		Seems & to be & good\\
%   		\hline
%   	\end{tabular}
%   	\caption{Random data for a table.}
% \end{table}

% Lorem ipsum dolor sit amet, consetetur sadipscing elitr, sed diam nonumy eirmod tempor invidunt ut labore et dolore magna aliquyam erat, sed diam voluptua. At vero eos et accusam et justo duo dolores et ea rebum. Stet clita kasd gubergren, no sea takimata sanctus est Lorem ipsum dolor sit amet. Lorem ipsum dolor sit amet, consetetur sadipscing elitr, sed diam nonumy eirmod tempor invidunt ut labore et dolore magna aliquyam erat, sed diam voluptua. At vero eos et accusam et justo duo dolores et ea rebum. Stet clita kasd gubergren, no sea takimata sanctus est Lorem ipsum dolor sit amet.


% \section{More Others}
% \subsection{What is calibration?}
% Here is an example of a matrix\cite{website:fermentas-lambda} in $A\in\mathcal{M}_n(\RR)$:
% $$
% A = 
% \begin{pmatrix}
% a_{11} & a_{12} & \ldots & a_{1n}\\
% a_{21} & \ddots & \ddots  & \vdots\\
% \vdots &  \ddots & \ddots  & \vdots\\
% a_{n1} &  \ldots &  \ldots & a_{1n}.
% \end{pmatrix}
% $$

% \subsection{Numerical methods for calibration}
% ...



\section{Legal, Social, Ethical and Professional Issues}

\subsection{Ethical and Professional Issues}
\noindent Research and projects within the medical domain are always inherently sensitive regardless of the kind of data involved or the presence of human participants. This sensitivity is amplified when a highly prominent industry stakeholder such as the NHS is interested, in view of the fact that it oversees public health across all of the UK. Right from the start, I have prioritized regular and transparent communication with our industry advisor through recurring meetings, while upholding implicit confidentiality agreements regarding the nature of the data and the project’s specific objectives. All analysis was conducted within the agreed-upon scope. All deliverables were presented in a coherent and actionable format, thereby reflecting my commitment to their distinct requirements and towards fostering a trustworthy working relationship.

\vspace{5pt}
\noindent Being cognizant of my social and ethical responsibility in this undertaking to advance public welfare, I have submitted an application in KCL’s Research Ethics Management Application System (REMAS) which should supplement the agreements and principles established at the time of inception of the project, considering that the project is a KEP with industry (NHS England). According to KCL and REMAS guidelines, this project is classed as “Minimal Risk”, in that it involves the study of pre-existing data that is not available to the general public, but is fully anonymous at the point which I as a researcher gain access to it. The industry advisor has kindly provided us the necessary data after complete anonymization, which removes any risk of personal identification. Still, I have opted to submit a comprehensive Full Application Form rather than a Minimal Risk Application to avoid any ambiguity in ethical review.

\vspace{5pt}
\noindent To further support this and in line with the guidelines listed in the General Data Protection Regulation (GDPR) as well as the Data Protection Act (DPA) 2018, the data was both shared with me and only accessed through secure organization / university credentials, meaning that it did not need to be fetched at all through any resource or API calls, eliminating the risk of interception. It was stored locally for on-machine data analysis and modelling through frequently used, open-source Python libraries.

\vspace{5pt}
\noindent Efforts have been taken to determine whether the project requires approval from any external entities, such as the Health Research Authority\cite{hraPlanning}. This was found to be not necessary. No recruitment of human participants was in the picture. Every care was taken to prevent any conflicts of interest from occurring, whether around other similar research, intellectual property, project objectives or any other sectors. I have also considered reliability measures to minimize the possibility of any kind of “reverse engineering” that may be carried out on my work. 

\subsubsection{Legal and Social Issues}
\noindent This project was conducted in strict adherence to the British Computer Society (BCS) Code of Conduct \cite{bcsCodeConduct}. The specific principle of “Public Interest” has been addressed by designing and developing a system to improve public health and patient safety. That of “Professional Competence and Integrity” has been demonstrated rigorous data handling protocols and a transparent approach and working practises. Socially, I strived to proactively mitigate any potential adverse effects by investigating any risk of algorithmic bias, to ensure a sensible outcome across all patient groups or demographics.

\vspace{5pt}	
\noindent I highlight and emphasize now that the research project is designed and intended as a \textbf{method of decision assistance, and not a decision making method.} The final accountability and responsibility remains with the concerned medical professional utilizing it. The study serves only to speed-up, scale-up, augment and refine clinical judgement, not to replace it. The research and model outputs are to be considered as one piece of evidence among many and should never be regarded as the deciding factor whatsoever in any situation or circumstance.

















	% A chapter gives a reasoned discussion about legal, social ethical and professional issues within the context of your project problem. You should also demonstrate that you are aware of the Code of Conduct \& Code of Good Practice issued by the British Computer Society (BSC) (\url{https://www.bcs.org/membership/become-a-member/bcs-code-of-conduct/}) for computer science project and Rule of Conduct issued by The Institution of Engineering and Technology (IET) (\url{https://www.theiet.org/about/governance/rules-of-conduct/}) for engineering project.  You should have applied their principles, where appropriate, as you carried out your project. You could consider aspects like: the effects of your project on the public well-being, security, software trustworthiness and risks, Intellectual Property and related issues, etc.
\section{Conclusion and Future Directions}
	\noindent This research project has successfully achieved all of its fundamental objectives, which were thoughtfully devised and aligned with Guy's \& St. Thomas' NHS Foundation Trust's operational interests and priorities at the time of project inception. The work provides GSTT with immediate tangible value and practical, actionable findings that are based on a very recent (i.e. derived from data ranging from 2024 to 2025) medically diverse patient population.

	\vspace{5pt}
	\noindent Within the given patient cohort and features in the provided dataset,  the analysis identified that Patient Age, Average Glycated Haemoglobin levels (HbA1c) and Estimated Glomerular Filtration Rate (eGFR) as the key factors contributing towards hypoglycaemia. It found that the key "decision points" where hypoglycaemia risk grew dramatically were at critically impaired renal function values (eGFR \textless= 25\%) which is medically classed as "Kidney Failure", and at HbA1c levels below 34mmol/mol.
	
	\vspace{5pt}
	\noindent The project serves as a foundation and proof-of-concept study for pre-emptive glycaemia management within the NHS. Its novelty stems from the analysis of a distinct, up-to-date dataset of diabetic patients admitted to London NHS wards. A key achievement of the project was the successful integration of disparate data streams from the GSTT's electronic patient health record system, Epic, that merged different aspects of patient data into a single cohesive dataset. This analysis acts as a blueprint that showcases how different Trusts can harness their own unique patient data to generate their own set of clinical actionable intelligence, and that the end-to-end approach I have used - from data exploration to model development to risk score creation - can either be customised by each Trust to accomplish their own goals or modified into a more general strategy to be adopted by various other Trusts across the NHS.

\subsection{Applicability within GSTT}
	\noindent In fulfilling all its goals, this project has confirmed the viability of applying machine learning to both resolve community healthcare issues in the NHS as well as extract new insights about patient and public health trends in the process. There are many ways GSTT or other Trusts within the NHS could apply the findings from this research into their healthcare workflows:

	\begin{itemize}
		\item \textbf{Integration into electronic health records systems like Epic:} The decision points found or even the risk score itself can be integrated directly into the Epic system. Upon admission of a new patient, the system can automatically decide the risk of hypoglycaemia based on these key factors and take necessary action (such as flagging the patient to be at greater risk of hypoglycaemia), whether that be assigning special rules to the patient such as a dedicated hypoglycaemia ward to be admitted in, or a review of current patient medication.
		\item \textbf{Personalised patient education and enhanced post-discharge support: } For instance, elderly high-risk patients can receive personalised instructions or increased follow-ups which can be decided by the risk score.
		\item \textbf{Benchmarking dashboard: } GSTT can leverage the key factors and risk score to create a monitoring and benchmarking dashboard to oversee patient numbers and track resources being expended, seasonal or yearly trends and so on. 
		\item \textbf{Universal record structures : } Since data had to be aggregated from different sources in order to be analysed, GSTT could devise a single universal way of storing records that would make it much easier to analyse, spurring further research.
		\item \textbf{Root cause analysis and simulation; } With the key factors now identified, these can be applied to past data to pull together the most likely causes for episodes in the past. They can also be used to simulate how or on what basis episodes could occur in other patient cohorts with their own distinct properties.
	\end{itemize}

%%%%% References
\bibliographystyle{ieeetr}
\bibliography{contents/sample1} 
%%%%% Declaration
%\include{structure/declaration}

%%%%% Appendix 
%Insert more section if necessary if more appendix sections are needed
%Remove or remark the following two lines if appendix is not necessary
\appendix
\include{contents/app_1}


%%%%%%%%%%%%%%%%%%%%%%%%%%%%%%%%%%%%%%%%%%%%%%%%%%%%
%%%%%%%%%%%%%%%%%%%%%%%%%%%%%%%%%%%%%%%%%%%%%%%%%%%%
%%%%%%%%%%%%%%%%%%%%%%%%%%%%%%%%%%%%%%%%%%%%%%%%%%%%
%%%%%%%%%%%%%%%%%%%%%%%%%%%%%%%%%%%%%%%%%%%%%%%%%%%%


\end{document}
