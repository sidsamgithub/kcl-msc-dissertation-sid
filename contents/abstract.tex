\red{The content of ``Abstract'' is in ``{\textbackslash}contents{\textbackslash}abstract.tex''}

\section*{Abstract}

% \begin{flushleft}

    \textbf{\underline{Project Variant:}} \textbf{ Variant 4 - Develop a weighted score and design score to predict risk of a hypoglycaemic episode before it occurs.} 

    \vspace{10pt}
    \par{It is well known that hypoglycemia as well as hyperglycemia are common adverse events in patients who receive blood sugar control medication, and they are also one of the most frequently cited causes of hospital admissions in people with diabetes. National quality improvement programmes from the Healthcare Quality Improvement Partnership (HQIP) and reviews of ambulance call-out data have shown that \textbf{\textit{lack of awareness}} by both affected individuals and their attendants is associated with a dramatically increased rate of complications, amongst other factors. Guy's and St. Thomas' NHS Foundation Trust (hereonwards referred to as GSTT) has found, after departmental investigation, that hypoglycemic episodes (also called "hypos") have been occurring unusually often, and seek to take measures to resolve such problems in a preventive manner as opposed to a corrective one. This research project has been undertaken in close collaboration with GSTT, one of the largest NHS trusts in the UK and an indispensable element of London's healthcare system, with almost 24000 staff across 5 major hospitals, handling over 3 million patients a year and generating an annual turnover of over \pounds3 billion.}

    \vspace{10pt}
    \par{This analytical study serves as a foundation and proof-of-concept to aid GSTT in pre-emptively reducing hypoglycemia within hospitalised inpatients, by utilising statistics \& machine learning techniques. Through exploratory data analysis I draw out relevant conclusions about the dataset around patient age, ethnicity and 
    
    I go on to identify the significant factors responsible for hypoglycemia within the dataset provided by the industry advisor from GSTT through exploratory data analysis, while also . I explore how they can be utilized to devise a risk score, to classify patients based on their risk of hypoglycemia. 
    
    HbA1c values identified as risky - 56 or so 
    eGFR 
    ethnicity
    major is glucose value 
    
 In conclusion, exhibit my findings with potential ways of applying them in practise in hospitals.}

    \vspace{10pt}

    \par{}

% \end{flushleft}

% \begin{itemize}
% 	\item A brief introduction to the project objectives
% 	\item A brief description of the main work of the project
% 	\item A brief description of the contributions, major findings, results achieved and principal conclusion of the project
% \end{itemize}
