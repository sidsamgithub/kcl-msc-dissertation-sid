\red{The content of ``Abstract'' is in ``{\textbackslash}contents{\textbackslash}abstract.tex''}

\section*{Abstract}

% \begin{flushleft}

    \textbf{\underline{Project Variant:}} \textbf{ Variant 4 - Develop a weighted score and design score to predict risk of a hypoglycaemic episode before it occurs.} 

    \vspace{10pt}
    \par{It is well known that hypoglycemia as well as hyperglycemia are common adverse events in patients who are on blood sugar control medication, and they are also one of the most frequently cited causes of hospital admissions in people with diabetes. National quality improvement programmes from the Healthcare Quality Improvement Partnership (HQIP) and reviews of ambulance call-outs have shown that \textbf{\textit{lack of awareness}} by both patients and their attendants is associated with a dramatically increased rate of complications, amongst other factors. Guy's and St. Thomas' NHS Foundation Trust (GSTT) have found, after departmental investigation, that hypoglycemic episodes (also called "hypos") }

    \vspace{10pt}
    \par{This analytical study serves as a foundation and proof-of-concept to aid GSTT in pre-emptively reducing hypoglycemia and its episodes within hospitalised patients, by utilising statistics \& machine learning techniques. I go on to identify the significant factors responsible for hypoglycemia within the dataset provided, and explore how they can be utilized to devise a risk score, to classify patients based on their risk of hypoglycemia, and in conclusion, exhibit my findings with potential ways of applying them in practise in hospitals.}

    \vspace{10pt}

    \par{}

% \end{flushleft}

% \begin{itemize}
% 	\item A brief introduction to the project objectives
% 	\item A brief description of the main work of the project
% 	\item A brief description of the contributions, major findings, results achieved and principal conclusion of the project
% \end{itemize}
