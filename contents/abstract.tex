% \red{The content of ``Abstract'' is in ``{\textbackslash}contents{\textbackslash}abstract.tex''}

\section*{Abstract}

% \begin{flushleft}

    \textbf{\underline{Project Variant:}} \textbf{ Variant 4 - Develop a weighted score and design score to predict risk of a hypoglycaemic episode before it occurs.} 

    \vspace{10pt}
    \par{ \noindent \textbf{Background:} It is well known that hypoglycemia as well as hyperglycemia are common adverse events in patients who receive blood sugar control medication, and they are also one of the most frequently cited causes of hospital admissions in people with diabetes. National quality improvement programmes from the Healthcare Quality Improvement Partnership (HQIP) and the study of ambulance call-out data have shown that \textbf{\textit{lack of awareness}} by both affected individuals and their attendants is associated with a dramatically increased rate of complications, amongst other factors. Guy's \& St. Thomas' NHS Foundation Trust (hereafter referred to as GSTT) has found, after careful deliberation and departmental review, that hypoglycemic episodes have been occurring with unusual frequency. The Trust now seeks to take measures to resolve such problems with a greater focus on prevention combined with early corrective action. This research project has been undertaken in close collaboration with GSTT, one of the largest NHS trusts in the UK and an indispensable element of London's healthcare system, with almost 24000 staff across 5 major hospitals, handling over 3 million patients a year and generating an annual turnover of over \pounds3 billion.}

    \vspace{10pt}
    \par{ \noindent \textbf{Methods:} I have conducted a retrospective analytical study using inpatient data provisioned by GSTT covering a period of one year, from mid 2024 to mid 2025. Exploratory data analysis was performed to derive insights related to trends in hypoglycaemic events across patient demographics including age and ethnicity, and hospital wards. Subsequently, machine learning and statistical techniques have been applied to find the key "decision points" for predictors to ascertain the conditions when hypoglycaemia risk increases dramatically. Decision Trees, Random Forests and Logistic Regression were the algorithms implemented and comparatively evaluated as part of the modelling process, which was based on preprocessed, unadulterated data to yield results that were meaningful and reflective of real-world conditions. XGBoost served as the main predictive model, and was trained on a conditionally sampled dataset drafted in a such a way so as to mitigate the imbalance between classes.
    
    \vspace{10pt}
    \par{ \noindent \textbf{Key Results:} Estimated Glomerular Filtration Rate (eGFR), Glycated Haemoglobin levels (HbA1c) and Age were determined to be the key predictors for hypoglycaemia. HbA1c levels below 34 were found to be the most significant contributing factor towards hypoglycaemic episodes, followed by critically low renal function (eGFR) levels below 25 percent. XGBOOST and LR 

    \vspace{10pt}
    \par{ \noindent \textbf{Conclusion:}   This analytical study serves as a foundation and proof-of-concept to aid GSTT in pre-emptively reducing hypoglycemia within hospitalised inpatients, by utilising statistics \& machine learning techniques. The study confirms that in the provided patient cohort, patient age, kidney function and past average blood sugar levels were the most significant factors to influence hypoglycaemia. The findings provide a thorough evidence base for GSTT to perform targeted actions or interventions, such as additional monitoring for elderly wards or enhanced staff training to take anticipatory remediative measures. This research promotes the standardisation of data collection across additional Trusts, while also laying the groundwork for scaling similar initiatives throughout the NHS Trust healthcare system.}


% \end{flushleft}

% \begin{itemize}
% 	\item A brief introduction to the project objectives
% 	\item A brief description of the main work of the project
% 	\item A brief description of the contributions, major findings, results achieved and principal conclusion of the project
% \end{itemize}
