\section{Legal, Social, Ethical and Professional Issues}

Research and projects within the medical domain are always inherently sensitive regardless of the kind of data involved or the presence of human participants. This sensitivity is amplified when a highly prominent industry stakeholder such as the NHS is interested, in view of the fact that it oversees public health across all of the UK. Right from the start, I have prioritized regular and transparent communication with our industry advisor through recurring meetings, while upholding implicit confidentiality agreements regarding the nature of the data and the project’s specific objectives. All analysis was conducted within the agreed-upon scope. All deliverables were presented in a coherent and actionable format, thereby reflecting my commitment to their distinct requirements and towards fostering a trustworthy working relationship.
Being cognizant of my social and ethical responsibility in this undertaking to advance public welfare, I have submitted an application in KCL’s Research Ethics Management Application System (REMAS) which should supplement the agreements and principles established at the time of inception of the project, considering that the project is a KEP with industry (NHS England). According to KCL and REMAS guidelines, this project is classed as “Minimal Risk” [ref], in that it involves the study of pre-existing data that is not available to the general public, but is fully anonymous at the point which I as a researcher gain access to it. The industry advisor has kindly provided us the necessary data after complete anonymization, which removes any risk of personal identification. (Still submitted a Full Application Form instead of a minimal risk application)
To further support this and in line with the guidelines listed in the General Data Protection Regulation (GDPR) as well as the Data Protection Act (DPA) 2018, the data was both shared with me and only accessed through secure organization / university credentials, meaning that it did not need to be fetched at all through any resource or API calls, eliminating the risk of interception. It was stored locally for on-machine data analysis and modelling through frequently used, open-source Python libraries, without the involvement of any online tools where the data has to be uploaded for research.
Efforts have been taken to determine whether the project requires approval from any external entities, for example the Health Research Authority[ref] https://www.hra.nhs.uk/planning-and-improving-research/research-planning/student-research/. This was found to be not necessary. No recruitment of human participants was in the picture. Every care was taken to prevent any conflicts of interest from occurring, whether around other similar research, intellectual property, project objectives or any other sectors. I have also considered reliability measures to minimize the possibility of any kind of “reverse engineering” that may be carried out on my work. 
This substantiates that I have displayed special adherence to the British Computer Society (BCS) Code of Conduct and Code of Practise[ref], especially the directives regarding “Public Interest” and “Professional Competence and Integrity”.

Socially, care has been taken to ensure that no adverse effects can occur as a result of this research















	% A chapter gives a reasoned discussion about legal, social ethical and professional issues within the context of your project problem. You should also demonstrate that you are aware of the Code of Conduct \& Code of Good Practice issued by the British Computer Society (BSC) (\url{https://www.bcs.org/membership/become-a-member/bcs-code-of-conduct/}) for computer science project and Rule of Conduct issued by The Institution of Engineering and Technology (IET) (\url{https://www.theiet.org/about/governance/rules-of-conduct/}) for engineering project.  You should have applied their principles, where appropriate, as you carried out your project. You could consider aspects like: the effects of your project on the public well-being, security, software trustworthiness and risks, Intellectual Property and related issues, etc.