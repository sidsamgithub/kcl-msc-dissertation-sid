\section{Introduction}
% It provides the background and context of the work.

\begin{flushleft}
Hypoglycemia is a condition that occurs when the human body’s blood glucose (sugar) level drops below the normal healthy range of 4.0 to 6.0 mmol/L. For the purposes of this project, severe hypoglycemia has been defined as a blood glucose of 2.2 mmol/L or lower. With glucose being the body’s main source of energy, hypoglycemia is a concern as it can disturb brain and bodily functions.  \\ \vspace{5pt}

Hypoglycemia is common in diabetics, especially those taking insulin, but it can occur in non-diabetics as well. Symptoms are rapid and successive, including feeling dizzy or sweating, shaking, feeling tired or weak and disorientated, unable to find one's bearing. \\ \vspace{5pt}

What is especially significant though, besides quick to appear symptoms, is that hypoglycemic episodes are dreadfully anxiety provoking, with practically every patient's main concern being if people around them would have the proper awareness of the necessary measures to take to resolve an episode should one occur. Even though remedial action is relatively simple (to provide the patient with sugary foods / liquids or solutions to restore blood glucose), it can lead to loss of consciousness \& memory, seizures, cardiac arrhythmias and even death if not taken speedily.    \\ \vspace{5pt}

Across GSTT, there are 0.5 million point of care glucose tests (POCT) carried out annually. GSTT possesses blood glucose / ketone data over the course of 1 year (Apr 2022-2023) with additional linked data including demographics, dates of admission and discharge, 
measurements including weight, blood pressure, HbA1c, renal function \& POCT. There is another kind of record they use called an ‘inpatient record’, and it was identified that the inpatient record misses approximately 10\% of glucose data due to operator error, as compared to the POCT glucose feed which contains more data. We would like to resolve this, while also analysing the available data in order to assist GSTT in detecting blood sugar related conditions, making measureable improvements to diabetic healthcare. \\ \vspace{5pt}

As evidenced by the National Diabetes Inpatient Safety Audit (NDISA) \hl{[insert reference]}, it has been recognized that severe hypoglycemia (blood glucose levels  2.2mmol/l) and recurrent severe hypoglycemia have been occurring relatively frequently at Guy’s and St. Thomas Hospitals (here onwards referred to as GSTT). The NDISA forms part of the National Database Audit (NDA), and it maintains that “The prevalence of diabetes continues to increase. In England between 2017-18 and 2021-22 prevalence of type 1 diabetes went up from 248,240 to 270,935 and the prevalence of type 2 and other diabetes from 2,952,695 to 3,336,980”, as of 2022. 

\end{flushleft}

\subsection{Aims and Objectives} 
% The problems and project objectives should be stated comprehensively. The motivations of the project should be presented. The techniques and approaches used to deal with the problem should be stated with justifications, and the contributions and main results achieved should be stated clearly. The structure of the report can be described briefly at the end see \autoref{sub:background}.



\subsubsection{Dissertation Length}
	The dissertation should be less than 15000 words.
	
\paragraph{Dissertation Length 2} ~\newline
	Refer to KEATS for suggested structure
	
\noindent \subparagraph{More subsections}~\newline
	
	

\subsection{Background and Literature Survey} \label{sub:background}
 It gives an overall picture about the work with a clear review of the relevant literature.  The background of the project should be given.  What have been done to deal with the problem should be stated clearly.  The pros and cons of various existing algorithms and approaches should be stated as well.  Differences between your proposed method and the existing ones should be briefly described. It is important to make sure that the discussion is structured and coherent; the key issues are summarised; key and relevant references are used critically analysed and the literature is covered comprehensively.

The following links may help on literature review:
\begin{itemize}
	\item \textbf{IEEE Xplore digital library} (\hyperref[http://ieeexplore.ieee.org]{http://ieeexplore.ieee.org/}): a resource for accessing IEEE published scientific and technical publications (You must be with King's network to get access to the digital library)
	\item \textbf{ScienceDirect.com} (\textbf{ScienceDirect.com} \hyperref[http://scienceDirect.com]{http://scienceDirect.com}): an electronic database offering journal papers not published by IEEE (You must be with King's network to get access to the database)
\end{itemize}

\subsection{Insert More Sections if Necessary}