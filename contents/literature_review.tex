% \red{The content of ``Background'' is in ``{\textbackslash}contents{\textbackslash}background.tex''}

\section{Literature Survey \& Review} 

\subsection{The Diabetes Management Tightrope}
% \vspace{10pt}
\noindent Managing diabetes often draws parallels with walking a metabolic tightrope. On one side lies the danger of hyperglycaemia and its associated \textit{long-term complications} such as diabetic retinopathy(damage to blood vessels in the eye leading to blindness) or renal function impairment (diabetic nephropathy), while on the other lies the \textit{immediate peril of hypoglycaemia}. For the longest time, clinicians have helped patients navigate this delicate balance, by utilising methods or practises that show where they are, but not necessarily where they are going, in terms of blood glucose measurements. Such a reactive approach with little account for anticipative elements has made hypoglycaemic episodes an unavoidable consequence of striving for tight glycaemic control. 

\vspace{5pt}
\noindent This "tightrope" extends beyond just taking efforts to stay safe, requiring diabetics and patients to perform constant risk assessments in everyday life. UK driving law from the DVLA mandates a blood glucose reading of above 5.0mmol/L with a repeat test every two hours for longer journeys in order to be considered safe to drive. Patients are required to carry a "hypo kit" with fast-acting glucose at all times. The worry about having episodes in public and having to rely on the awareness of strangers or being a hindrance to social situations is a constant concern. As discussed \hyperref[sec:clinicalOverview]{here previously} this also places notable financial strain on NHS resources through emergency or ambulance costs.

\vspace{5pt}
\noindent With scientific and technological progress that inevitably comes with time, comes the promise of a potential safety net: the ability to anticipate or foresee signs of an episode before they present. The dangerous tightrope walk can then be transformed into a manageable path, with the help of predictive systems built with advanced sensing technologies and computational power, that can assist patients to take pre-emptive action. This review will chart the progress in this field, understanding the methodologies and ideas that have been applied to forecast hypoglycaemic events, from early tracking methods to highly optimized modern algorithms.




\subsection{Methods Of Monitoring Blood Glucose Over Time}




















\begin{flushleft}
Various kinds of different prediction models have already been devised and developed for predicting hypoglycemia. Yi Wu and others have systematically compared, and evaluated the applicability of models in clinical practice in a paper in Biological Research for Nursing[1] 
where it was found that the major predictors were age, HbA1c, history of hypoglycemia, and insulin use. Lin Yang, Zhiguang Zhou have carried out similar research in the Frontiers in Public Health journal[2] uncovering risk factors that could possibly lead to hypoglycemic 
events, after employing various data driven models based on ML techniques such as neural networks, autoregressive / ensemble learning and such. \\ \vspace{5pt}

In silico proof of concept studies like the one from Zecchin[3] have also been researched to investigate how continuous glucose monitoring short-term glucose prediction algorithms could be exploited to recognise the run up to hypoglycemic episodes, allowing the patient to 
take appropriate countermeasures to mitigate events. They found that there was a significant reduction in both the time spent in a hypoglycemic event as well as the number 
of hypoglycemic events. \\ \vspace{5pt}

As this is a Knowledge Exchange Project (KEP) with NHS England I have been provided a real world dataset from GSTT. Medical data is difficult to obtain, and it rarely fits a research objective without needing much modification. 

\end{flushleft}