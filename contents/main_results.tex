\section{Main Results and Findings}\label{sec:mainResults}

\noindent \textbf{Note: Primarily due to the nature of exploratory data analysis and because of the quantity of research questions stated by GSTT, it was not possible to fit all the exploratory and modelling research done into the report because it would have certainly exceeded the word limit of 15000 words, while preventing me from explaining every salient point in the visualisations. Additionally even if made to fit within the word limit, it would have caused the report to explode in the number of pages making it difficult to maintain the reader's understanding. My code submission which is a python notebook (.ipynb file) is meant for exhibiting findings along with their supporting material, and I have documented many notes explaining the rationale behind every decision and the interpretation of every model. Please have a look through the python notebook in PyCharm or any IDE with Python and Jupyter to see the whole set of findings, the video demonstration will prove this.}

\subsection{Initial Data Pre-processing}
\noindent Raw data provided was first combined into a single file and then read into a Pandas dataframe. As part of initial cleanup, 2164 duplicate rows in the dataset were identified and dropped, as shown in \autoref{fig:dropDuplicates}. 

\begin{figure}[H]
		\centering
		\includegraphics[scale=0.8]{figures/python_code/drop_duplicate_rows.png}
		\caption{Dropping duplicate rows}
		\label{fig:dropDuplicates}
\end{figure}

\noindent Following this, the percentage of missing values was evaluated prior to carrying out feature engineering. \autoref{fig:missingValuePercentages} displays the percentage of missing data across all raw columns:

\begin{figure}[H]
		\centering
		\includegraphics[scale=0.4]{figures/python_code/percent_missing_values_in_every_column.png}
		\caption{Percentage of missing values in each column}
		\label{fig:missingValuePercentages}
\end{figure}

\noindent The column Admit Weight was rendered unusable here, because it was missing 97\% of its content. Weight values cannot be calculated or imputed accurately based on the only available related features that were Age and Ethnicity, as Weight is a unique characteristic value of a patient just like Height and BMI, and to use class-based (ethnicity and age based) averages would \textbf{mislead the analysis} and produce unreliable results, so Weight is not considered in any further study hereonwards.

\subsubsection{Feature Extraction}
\noindent Next, all the columns were thoroughly examined for the exact relevant numerical values (for instance, age was given as "57kg" in the raw data as depicted in \autoref{fig:rawDataset}.) and new columns were created to store only the extracted numerical values for analysis. After this point, the usable data resembled the structure of the cleaned data as shown in \autoref{fig:cleanedDataset}.

\subsubsection{Feature Engineering}
\noindent After having acquired clear numerical values from the raw data through the newly extracted features, additional derived features were formulated based on these values. The resulting features are listed in \autoref{sec:derivedData} along with the criteria and logic used in their construction.   

\subsection{Exploratory Data Analysis and Interpreting Model Results}

\noindent \textbf{Note: Primarily due to the nature of exploratory data analysis and because of the quantity of research questions stated by GSTT, it was not possible to fit all the exploratory and modelling research done into the report because it would have certainly exceeded the word limit of 15000 words, while preventing me from explaining every salient point in the visualisations. Additionally even if made to fit within the word limit, it would have caused the report to explode in the number of pages making it difficult to maintain the reader's understanding. My code submission which is a python notebook (.ipynb file) is meant for exhibiting findings along with their supporting material, and I have documented many notes explaining the rationale behind every decision and the interpretation of every model. Please have a look through the python notebook in PyCharm or any IDE with Python and Jupyter to see the whole set of findings, the video demonstration will prove this.}

% \vspace{5pt}
% \noindent GSTT had outlined several major research questions that they sought to understand through this analysis. The questions were shaped by their operational priorities and clinical needs, aiming to discover essential details that could improve as well as guide patient care strategies, and refine medical procedures. I have structured the subsequent analysis to address these focal points through visualisation techniques.

% \subsubsection{How many patients are in each ward, and which wards do patients spend the most time in on average?}
% \noindent GSTT wanted to examine patient flow within wards, focusing on ward assignments and length of stay. \textbf{The top 5 wards to house patients were found to be: } William Gull Ward (4773), St. Thomas Admissions Ward (2997), St. Thomas Albert Ward (2446), St. Thomas Anne ward(2062), St. Thomas Alexandra Ward(1715).  

% \noindent \begin{figure}[H]
% 		\centering
% 		\includegraphics[scale=0.5]{figures/python_code/patients_in_each_ward2.png}
% 		\caption{Number of patients in each ward}
% 		\label{fig:NumOfPatientsInWards}
% \end{figure}


% \subsection{Interpreting model results}

% \subsubsection{Decision Trees}



% \section{Math equations}
% % \textcolor{red}{This section is for demonstration of equations, figures, tables, which is not required for the report.}
% \subsection{Maths}
% \begin{equation}\label{eq:BS}
% \frac{\D S_t}{S_t} = r \D t + \sigma \D W_t,
% \qquad S_0>0,
% \end{equation}

% The equation $\sigma = m a$ follows easily~\cite{Doe11}.


% \subsection{Glossary and acronyms}

% \newglossaryentry{Linux}
% {
%     name=latexlinux,
%     description={Is a markup language specially suited for 
% scientific documents}
% }

% \newglossaryentry{lvm}
% {
%     name=lvmformula,
%     description={A mathematical expression}
% }

% \Glspl{Linux} and other Unix operating systems are better then Windows because they support \gls{lvm} out of the box~\cite{Joh11}\insertref{A ref is missing here}. 

% \subsection{Figures}
% Here is an example~\cite{JohSil05} of how to insert a picture:

% \begin{figure}[!ht]
% \centering
% \subfigure{\includegraphics[scale=0.2]{figures/Picture.eps}}
% \caption{This is the caption for the figure.}
% \label{fig:Pict}
% \end{figure}


% \begin{figure}[!ht]
% \centering
% \missingfigure{If you know there will be a figure, but you still need to create it.}
% \caption{This is the caption for the figure which is not even present.}
% \label{fig:PictMis}
% \end{figure}


% Lorem ipsum dolor sit amet, consetetur sadipscing elitr, sed diam nonumy eirmod tempor invidunt ut labore et dolore magna aliquyam erat, sed diam voluptua. At vero eos et accusam et justo duo dolores et ea rebum. Stet clita kasd gubergren, no sea takimata sanctus est Lorem ipsum dolor sit amet. Lorem ipsum dolor sit amet, consetetur sadipscing elitr, sed diam nonumy eirmod tempor invidunt ut labore et dolore magna aliquyam erat, sed diam voluptua. At vero eos et accusam et justo duo dolores et ea rebum. Stet clita kasd gubergren, no sea takimata sanctus est Lorem ipsum dolor sit amet.\todo{This is a small Todo, please take care!}

% or two side-by-side pictures:

% \begin{figure}[!ht]
% \centering
% \subfigure{\includegraphics[scale=0.3]{figures/Picture.eps}}
% \hspace{15pt}
% \subfigure{\includegraphics[scale=0.3]{figures/Picture.eps}}

% \caption{Another caption}
% \label{fig:Pict2}
% \end{figure}



% \subsection{Table}
% Lorem ipsum dolor sit amet, consetetur sadipscing elitr, sed diam nonumy eirmod tempor invidunt ut labore et dolore magna aliquyam erat, sed diam voluptua. At vero eos et accusam et justo duo dolores et ea rebum. Stet clita kasd gubergren, no sea takimata sanctus est Lorem ipsum dolor sit amet. Lorem ipsum dolor sit amet, consetetur sadipscing elitr, sed diam nonumy eirmod tempor invidunt ut labore et dolore magna aliquyam erat, sed diam voluptua. At vero eos et accusam et justo duo dolores et ea rebum. Stet clita kasd gubergren, no sea takimata sanctus est Lorem ipsum dolor sit amet\explainindetail{This needs further explanation}.
% \begin{table}[!ht]
% 	\centering
% 	\begin{tabular}{|l|rl|}
% 		\hline
% 		Something & Someother & Thing \\
%   		Seems & to be & good\\
%   		\hline
%   	\end{tabular}
%   	\caption{Random data for a table.}
% \end{table}

% Lorem ipsum dolor sit amet, consetetur sadipscing elitr, sed diam nonumy eirmod tempor invidunt ut labore et dolore magna aliquyam erat, sed diam voluptua. At vero eos et accusam et justo duo dolores et ea rebum. Stet clita kasd gubergren, no sea takimata sanctus est Lorem ipsum dolor sit amet. Lorem ipsum dolor sit amet, consetetur sadipscing elitr, sed diam nonumy eirmod tempor invidunt ut labore et dolore magna aliquyam erat, sed diam voluptua. At vero eos et accusam et justo duo dolores et ea rebum. Stet clita kasd gubergren, no sea takimata sanctus est Lorem ipsum dolor sit amet.


% \section{More Others}
% \subsection{What is calibration?}
% Here is an example of a matrix\cite{website:fermentas-lambda} in $A\in\mathcal{M}_n(\RR)$:
% $$
% A = 
% \begin{pmatrix}
% a_{11} & a_{12} & \ldots & a_{1n}\\
% a_{21} & \ddots & \ddots  & \vdots\\
% \vdots &  \ddots & \ddots  & \vdots\\
% a_{n1} &  \ldots &  \ldots & a_{1n}.
% \end{pmatrix}
% $$

% \subsection{Numerical methods for calibration}
% ...


