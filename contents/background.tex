\red{The content of ``Background'' is in ``{\textbackslash}contents{\textbackslash}background.tex''}

\section{Background theories \& Literature Survey} 

\begin{flushleft}
Various kinds of different prediction models have already been devised and developed for predicting hypoglycemia. Yi Wu and others have systematically compared, and evaluated the applicability of models in clinical practice in a paper in Biological Research for Nursing[1] 
where it was found that the major predictors were age, HbA1c, history of hypoglycemia, and insulin use. Lin Yang, Zhiguang Zhou have carried out similar research in the Frontiers in Public Health journal[2] uncovering risk factors that could possibly lead to hypoglycemic 
events, after employing various data driven models based on ML techniques such as neural networks, autoregressive / ensemble learning and such. \\ \vspace{5pt}

In silico proof of concept studies like the one from Zecchin[3] have also been researched to investigate how continuous glucose monitoring short-term glucose prediction algorithms could be exploited to recognise the run up to hypoglycemic episodes, allowing the patient to 
take appropriate countermeasures to mitigate events. They found that there was a significant reduction in both the time spent in a hypoglycemic event as well as the number 
of hypoglycemic events. \\ \vspace{5pt}

As this is a Knowledge Exchange Project (KEP) with NHS England I have been provided a real world dataset from GSTT. Medical data is difficult to obtain, and it rarely fits a research objective without needing much modification. 

% In this project I first go on to perform exploratory data analysis to better understand how the data is structured, how it is sourced, and how it can be visualised to extract maximum information.

\end{flushleft}